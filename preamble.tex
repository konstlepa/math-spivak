\usepackage[utf8]{inputenc}
\usepackage[english]{babel}
\usepackage{xcolor}
\usepackage{enumitem}
\usepackage{mdframed}
\usepackage{etoolbox}
\usepackage[hidelinks]{hyperref}
\usepackage{hyperref}
\usepackage{thmtools}
\usepackage{thm-restate}
\usepackage{stackengine}
\usepackage[active,tightpage]{preview}
\usepackage{tikz}
\usepackage{graphicx}
\graphicspath{{figs/}}

\mdfsetup{
	innerleftmargin=6pt,
	innerrightmargin=6pt,
	splittopskip=\topskip,
	skipabove=1.5em,
	skipbelow=1.5em
}

\makeatletter
\newcommand\xlabel[2][]{\phantomsection\def\@currentlabelname{#1}\label{#2}}
\makeatother

\renewcommand{\PreviewBorder}{0.5em}

%% Restore par-indent for preview.
\edef\keptparindent{\the\parindent}
\patchcmd{\preview}
{\ignorespaces} %%% \preview ends with \ignorespaces
{\parindent\keptparindent\ignorespaces}
{}{}

\newcommand{\Newpage}{\end{preview}\begin{preview}}

\makeatletter
\def\p@section{\thechapter.}
\makeatother

\setlist[1]{leftmargin=*,labelindent=\parindent}
\setlist[2]{leftmargin=*,labelindent=\parindent}

\newlist{bycases}{enumerate}{2}
\setlist[bycases,1]{label=Case\,\arabic{bycasesi}., ref=\arabic{bycasesi}}
\setlist[bycases,2]{label=Case\,\arabic{bycasesi}\,\alph{bycasesii}., ref=\arabic{bycasesi}.\alph{bycasesii},labelindent=1em,leftmargin=!}

\newlist{lenumerate}{enumerate}{2}
\setlist[lenumerate,1]{label=(\alph{lenumeratei}), ref=\alph{lenumeratei}}

\setlist[enumerate]{leftmargin=*}
\setlist[enumerate,1]{label=(\arabic*), ref=\arabic*}
\setlist[enumerate,2]{label=(\alph*), ref=\theenumi\,\alph*}

\renewcommand{\datename}{}

\declaretheoremstyle[
	headfont=\normalfont\bfseries,
	notefont=\normalfont\bfseries,
	notebraces={(}{)},
	bodyfont=\normalfont,
	postheadspace=0.5em,
	preheadhook=\begin{mdframed}, postfoothook=\end{mdframed}
]{problemstyle}

\declaretheoremstyle[
	headfont=\normalfont\bfseries,
	notefont=\normalfont\bfseries,
	notebraces={(}{)},
	bodyfont=\normalfont\itshape,
	postheadspace=0.5em,
	spaceabove=1.5em
]{theoremstyle}

\declaretheorem[style=problemstyle, numberwithin=section]{problem}
\declaretheorem[style=theoremstyle, numberwithin=section]{theorem}
\declaretheorem[style=theoremstyle, sibling=theorem]{lemma}
\declaretheorem[style=theoremstyle, sibling=theorem]{definition}
\declaretheorem[style=theoremstyle, sibling=theorem]{axiom}
\declaretheorem[style=theoremstyle, sibling=theorem]{corollary}

\renewcommand{\theproblem}{\thechapter-\arabic{problem}}
\renewcommand{\thetheorem}{\thechapter.\arabic{theorem}}
\renewcommand{\thelemma}{\thechapter.\arabic{lemma}}
\renewcommand{\thedefinition}{\thechapter.\arabic{definition}}
\renewcommand{\thecorollary}{\thechapter.\arabic{corollary}}
\renewcommand{\theaxiom}{\thechapter.\arabic{axiom}}

\declaretheoremstyle[
	spaceabove=6pt,
	spacebelow=6pt,
	headfont=\normalfont\itshape\bfseries,
	notefont=\normalfont\itshape\bfseries,
	notebraces={of }{},
	bodyfont=\normalfont,
	postheadspace=1em,
	qed=$\square$
]{proofstyle}
\let\proof\relax
\declaretheorem[style=proofstyle, unnumbered]{proof}

\newcommand{\bibspivak}{
	\bibitem{spivak}
	Michael Spivak,
	\emph{Calculus},
	4th ed.,
	Publish or Perish, Inc.,
	2008.
}

\newcommand{\TBD}{{\color{red} TBD}}
\newcommand{\varemph}{\itshape}

\newenvironment{notmine}[1][]{\color{gray}}{}
\newenvironment{done}[1][]{\color{gray}}{}
\newenvironment{changed}[1][]{\color{blue}}{}

\let\emph\relax
\DeclareTextFontCommand{\emph}{\bfseries}

\newcommand{\PartProof}[1]{\bigskip\emph{(\ref{#1})}\enspace}
\newcommand{\PartProblem}[1]{\bigskip\emph{(\ref{#1})}\enspace}
